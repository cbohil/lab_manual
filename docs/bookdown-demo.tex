% Options for packages loaded elsewhere
\PassOptionsToPackage{unicode}{hyperref}
\PassOptionsToPackage{hyphens}{url}
%
\documentclass[
]{book}
\usepackage{lmodern}
\usepackage{amssymb,amsmath}
\usepackage{ifxetex,ifluatex}
\ifnum 0\ifxetex 1\fi\ifluatex 1\fi=0 % if pdftex
  \usepackage[T1]{fontenc}
  \usepackage[utf8]{inputenc}
  \usepackage{textcomp} % provide euro and other symbols
\else % if luatex or xetex
  \usepackage{unicode-math}
  \defaultfontfeatures{Scale=MatchLowercase}
  \defaultfontfeatures[\rmfamily]{Ligatures=TeX,Scale=1}
\fi
% Use upquote if available, for straight quotes in verbatim environments
\IfFileExists{upquote.sty}{\usepackage{upquote}}{}
\IfFileExists{microtype.sty}{% use microtype if available
  \usepackage[]{microtype}
  \UseMicrotypeSet[protrusion]{basicmath} % disable protrusion for tt fonts
}{}
\makeatletter
\@ifundefined{KOMAClassName}{% if non-KOMA class
  \IfFileExists{parskip.sty}{%
    \usepackage{parskip}
  }{% else
    \setlength{\parindent}{0pt}
    \setlength{\parskip}{6pt plus 2pt minus 1pt}}
}{% if KOMA class
  \KOMAoptions{parskip=half}}
\makeatother
\usepackage{xcolor}
\IfFileExists{xurl.sty}{\usepackage{xurl}}{} % add URL line breaks if available
\IfFileExists{bookmark.sty}{\usepackage{bookmark}}{\usepackage{hyperref}}
\hypersetup{
  pdftitle={CDL Virtual Advisor (Lab Manual)},
  pdfauthor={Corey Bohil},
  hidelinks,
  pdfcreator={LaTeX via pandoc}}
\urlstyle{same} % disable monospaced font for URLs
\usepackage{color}
\usepackage{fancyvrb}
\newcommand{\VerbBar}{|}
\newcommand{\VERB}{\Verb[commandchars=\\\{\}]}
\DefineVerbatimEnvironment{Highlighting}{Verbatim}{commandchars=\\\{\}}
% Add ',fontsize=\small' for more characters per line
\usepackage{framed}
\definecolor{shadecolor}{RGB}{248,248,248}
\newenvironment{Shaded}{\begin{snugshade}}{\end{snugshade}}
\newcommand{\AlertTok}[1]{\textcolor[rgb]{0.94,0.16,0.16}{#1}}
\newcommand{\AnnotationTok}[1]{\textcolor[rgb]{0.56,0.35,0.01}{\textbf{\textit{#1}}}}
\newcommand{\AttributeTok}[1]{\textcolor[rgb]{0.77,0.63,0.00}{#1}}
\newcommand{\BaseNTok}[1]{\textcolor[rgb]{0.00,0.00,0.81}{#1}}
\newcommand{\BuiltInTok}[1]{#1}
\newcommand{\CharTok}[1]{\textcolor[rgb]{0.31,0.60,0.02}{#1}}
\newcommand{\CommentTok}[1]{\textcolor[rgb]{0.56,0.35,0.01}{\textit{#1}}}
\newcommand{\CommentVarTok}[1]{\textcolor[rgb]{0.56,0.35,0.01}{\textbf{\textit{#1}}}}
\newcommand{\ConstantTok}[1]{\textcolor[rgb]{0.00,0.00,0.00}{#1}}
\newcommand{\ControlFlowTok}[1]{\textcolor[rgb]{0.13,0.29,0.53}{\textbf{#1}}}
\newcommand{\DataTypeTok}[1]{\textcolor[rgb]{0.13,0.29,0.53}{#1}}
\newcommand{\DecValTok}[1]{\textcolor[rgb]{0.00,0.00,0.81}{#1}}
\newcommand{\DocumentationTok}[1]{\textcolor[rgb]{0.56,0.35,0.01}{\textbf{\textit{#1}}}}
\newcommand{\ErrorTok}[1]{\textcolor[rgb]{0.64,0.00,0.00}{\textbf{#1}}}
\newcommand{\ExtensionTok}[1]{#1}
\newcommand{\FloatTok}[1]{\textcolor[rgb]{0.00,0.00,0.81}{#1}}
\newcommand{\FunctionTok}[1]{\textcolor[rgb]{0.00,0.00,0.00}{#1}}
\newcommand{\ImportTok}[1]{#1}
\newcommand{\InformationTok}[1]{\textcolor[rgb]{0.56,0.35,0.01}{\textbf{\textit{#1}}}}
\newcommand{\KeywordTok}[1]{\textcolor[rgb]{0.13,0.29,0.53}{\textbf{#1}}}
\newcommand{\NormalTok}[1]{#1}
\newcommand{\OperatorTok}[1]{\textcolor[rgb]{0.81,0.36,0.00}{\textbf{#1}}}
\newcommand{\OtherTok}[1]{\textcolor[rgb]{0.56,0.35,0.01}{#1}}
\newcommand{\PreprocessorTok}[1]{\textcolor[rgb]{0.56,0.35,0.01}{\textit{#1}}}
\newcommand{\RegionMarkerTok}[1]{#1}
\newcommand{\SpecialCharTok}[1]{\textcolor[rgb]{0.00,0.00,0.00}{#1}}
\newcommand{\SpecialStringTok}[1]{\textcolor[rgb]{0.31,0.60,0.02}{#1}}
\newcommand{\StringTok}[1]{\textcolor[rgb]{0.31,0.60,0.02}{#1}}
\newcommand{\VariableTok}[1]{\textcolor[rgb]{0.00,0.00,0.00}{#1}}
\newcommand{\VerbatimStringTok}[1]{\textcolor[rgb]{0.31,0.60,0.02}{#1}}
\newcommand{\WarningTok}[1]{\textcolor[rgb]{0.56,0.35,0.01}{\textbf{\textit{#1}}}}
\usepackage{longtable,booktabs}
% Correct order of tables after \paragraph or \subparagraph
\usepackage{etoolbox}
\makeatletter
\patchcmd\longtable{\par}{\if@noskipsec\mbox{}\fi\par}{}{}
\makeatother
% Allow footnotes in longtable head/foot
\IfFileExists{footnotehyper.sty}{\usepackage{footnotehyper}}{\usepackage{footnote}}
\makesavenoteenv{longtable}
\usepackage{graphicx,grffile}
\makeatletter
\def\maxwidth{\ifdim\Gin@nat@width>\linewidth\linewidth\else\Gin@nat@width\fi}
\def\maxheight{\ifdim\Gin@nat@height>\textheight\textheight\else\Gin@nat@height\fi}
\makeatother
% Scale images if necessary, so that they will not overflow the page
% margins by default, and it is still possible to overwrite the defaults
% using explicit options in \includegraphics[width, height, ...]{}
\setkeys{Gin}{width=\maxwidth,height=\maxheight,keepaspectratio}
% Set default figure placement to htbp
\makeatletter
\def\fps@figure{htbp}
\makeatother
\setlength{\emergencystretch}{3em} % prevent overfull lines
\providecommand{\tightlist}{%
  \setlength{\itemsep}{0pt}\setlength{\parskip}{0pt}}
\setcounter{secnumdepth}{5}
\usepackage{booktabs}
\usepackage{amsthm}
\makeatletter
\def\thm@space@setup{%
  \thm@preskip=8pt plus 2pt minus 4pt
  \thm@postskip=\thm@preskip
}
\makeatother
\usepackage[]{natbib}
\bibliographystyle{apalike}

\title{CDL Virtual Advisor (Lab Manual)}
\author{Corey Bohil}
\date{2020-08-16}

\begin{document}
\maketitle

{
\setcounter{tocdepth}{1}
\tableofcontents
}
\hypertarget{overview}{%
\chapter{Overview}\label{overview}}

The goal of this book is to serve as a resource for procedures in the lab, tutorials on software and hardware use, and some other advice that I hope will be helpful (e.g., advice on scientific manuscript writing). My guess is that it will serve primarily to get people started in the lab, and also as a refresher on some details about data handling/analyis and use of some equipment.

This is a living document; it will never be ``finished''. We should update it regularly.

This is a \emph{sample} book written in \textbf{Markdown}. You can use anything that Pandoc's Markdown supports, e.g., a math equation \(a^2 + b^2 = c^2\).

The \textbf{bookdown} package can be installed from CRAN or Github:

\begin{Shaded}
\begin{Highlighting}[]
\KeywordTok{install.packages}\NormalTok{(}\StringTok{"bookdown"}\NormalTok{)}
\CommentTok{# or the development version}
\CommentTok{# devtools::install_github("rstudio/bookdown")}
\end{Highlighting}
\end{Shaded}

Remember each Rmd file contains one and only one chapter, and a chapter is defined by the first-level heading \texttt{\#}.

To compile this example to PDF, you need XeLaTeX. You are recommended to install TinyTeX (which includes XeLaTeX): \url{https://yihui.name/tinytex/}.

\hypertarget{getting_started}{%
\chapter{Getting Started in the Lab}\label{getting_started}}

Essential resources and steps

\hypertarget{paperwork}{%
\section{Paperwork}\label{paperwork}}

See chapter 7 - Students

OK NEED TO FIGURE OUT WHAT WILL GO IN THIS SECTION - UNDERGRAD STUFF HERE OR IN CH 7?

CK - All of it here?? That way it's together, and then ch 7 can be just for grad students and include things like program requirements, the checklist, etc.??

\hypertarget{undergraduate-students}{%
\subsection{Undergraduate Students}\label{undergraduate-students}}

There are two options for undergraduate RA's - register or sign the volunteer agreement. Each RA only needs to choose one or the other.

\hypertarget{registration}{%
\subsection{Registration}\label{registration}}

Registering for Directed Independent Research (PSY4912) has the advantage of putting research on their transcript. We almost always register them for 0 credit hours so they can avoid paying tuition/fees, but they can take more if they need to fill out their schedule to reach full-time, keep a scholarship, etc. The \href{docs/resources/Undergraduate\%20Registration\%20Agreement\%20Summer\%202020.pdf}{Restricted Registration Form} is filled out according to this template. IMPORTANT: Be sure the version you have is the most recent one (i.e., check with the program assistant, Mikala).

\begin{enumerate}
\def\labelenumi{\arabic{enumi}.}
\tightlist
\item
  Grad student fills out the term/course \# (PSY4912) and the ``Assignments and Expectations'' section. The latter is completely made up; only the dates matter. The due date of Assignment 1 must be before the end of Add/Drop, and Assignment 3 must be due before the end of finals week (the last day of classes is safe). All else can go unchanged.
\item
  Undergrad fills out the form with their information and signs the document (by hand or with an e-signer - NEED TO HEAR BACK FROM KAREN COX ON REQUIREMENTS FOR ESIGNER).
\item
  Grad student sends form to the instructor/PI (Dr.~Bohil) to be signed and then sends signed form back to the undergrad.
\item
  Undergrad sends completed and signed form to psych advising (room: PSY 250; email: \href{mailto:Psychadvising@ucf.edu}{\nolinkurl{Psychadvising@ucf.edu}})
  Due date: Signed form must be delivered to psych advising by the end of Add/Drop.
\end{enumerate}

\hypertarget{volunteer-services-agreement.}{%
\subsection{Volunteer Services Agreement.}\label{volunteer-services-agreement.}}

Students that opt not to register must sign the Volunteer Services Agreement. This brings them under the regulatory umbrella of UCF (i.e., liability insurance, ethics training compliance, etc.). See the \href{docs/resources/VolunteerFormInstructions.pdf}{volunteer form instructions} for guidance.
- Grad student initiates form here: \url{https://compliance.ucf.edu/enterprise-risk-management/university-volunteers/}
- Note: The faculty member (Dr.~Bohil) must be listed as the supervisor, not a grad student.
- Note: The forms previously were good for only one semester, as of Summer 2020 they are good for a year.

\hypertarget{graduate-students}{%
\subsection{Graduate Students}\label{graduate-students}}

\hypertarget{software}{%
\section{Software}\label{software}}

You can label chapter and section titles using \texttt{\{\#label\}} after them, e.g., we can reference Chapter \ref{intro}. If you do not manually label them, there will be automatic labels anyway, e.g., Chapter \ref{methods}.

\hypertarget{irb}{%
\section{IRB}\label{irb}}

\hypertarget{subject-recruitment}{%
\section{Subject Recruitment}\label{subject-recruitment}}

\hypertarget{sona}{%
\subsection{Sona}\label{sona}}

Recruitment is done through the SONA system - see \href{docs/resources/SONA\%20Guide.pdf}{using SONA} for a guide to setting up and running studies using the university subject pool.

The SONA system is administered by Dr.~Chrysalis Wright, and the primary point of contact (graduate student assistant) is Mark Crisafulli. Questions should be directed to Mr.~Crisafulli (\href{mailto:psych-research@ucf.edu}{\nolinkurl{psych-research@ucf.edu}}).

\hypertarget{general-advice}{%
\section{General advice}\label{general-advice}}

\begin{enumerate}
\def\labelenumi{\arabic{enumi}.}
\tightlist
\item
  Don't get attached to any experiment (or theory); just run as many as you can!
\end{enumerate}

\begin{enumerate}
\def\labelenumi{\roman{enumi})}
\tightlist
\item
  Expect to be surprised; your hypotheses will often be wrong.
\item
  We need to publish research, but desperately needing to publish something is a recipe
  for over-interpretation of results
\item
  You can mitigate this to some extent by running lots of studies. Think of it like
  drilling for oil; its good to have a lot of wells going at once.
\end{enumerate}

\hypertarget{modifying-this-lab-manual-grad-students-only}{%
\section{Modifying this Lab Manual (Grad students only)}\label{modifying-this-lab-manual-grad-students-only}}

\begin{enumerate}
\def\labelenumi{\arabic{enumi}.}
\item
  Install bookdown package in RStudio: \texttt{install.packages(\textquotesingle{}bookdown\textquotesingle{})}
\item
  clone the Lab Manual repo onto your computer \& associate the folder with a new R project in RSTudio (i.e., once you've copied the repo folder onto your computer, create a New Project in RStudio {[}with Git support{]} and assocciate it with the online repo URL as well as the folder on your computer that you just created (i.e., the clone of the online repo)). After that you should be able to open the project in RStudio, make changes and commit/push them to the online Github repo
\item
  Modify only the .Rmd files that correspond to chapters. If you need to add an image or file to link to, put it in the docs/resources/ folder. (e.g., see Sona, fNIRS examples in this book)
\item
  When finished, go to the Build tab (should be next to your Git tab) and select `Build Book'. If the book builds \& launches in a local browser, everything worked correctly.
\item
  Commit changes to Git using the Git tab, and push the changes to the Github repo
\end{enumerate}

Additional notes: \url{https://bookdown.org/yihui/bookdown/}

\hypertarget{essential-reading}{%
\chapter{Essential Reading}\label{essential-reading}}

Here is a list of papers, book chapters, or books that everyone in the lab should read.

\hypertarget{categorization}{%
\section{Categorization}\label{categorization}}

\begin{enumerate}
\def\labelenumi{\arabic{enumi}.}
\setcounter{enumi}{-1}
\tightlist
\item
  Signal detection theory (Swets book chapter; other intro chapters)
\item
  Human Category Learning. Ashby \& Maddox (2005). \url{https://www.annualreviews.org/doi/pdf/10.1146/annurev.psych.56.091103.070217}
\item
  Human Category Learning 2.0. \url{https://www.ncbi.nlm.nih.gov/pmc/articles/PMC3076539/}
\item
  Multidimensional Signal Detection Theory. \url{https://labs.psych.ucsb.edu/ashby/gregory/sites/labs.psych.ucsb.edu.ashby.gregory/files/pubs/ashbysotogrt2015.pdf}
\item
  General recognition theory with individual differences: a new method for examining perceptual and decisional interactions with an application to face perception. \url{https://link.springer.com/article/10.3758\%2Fs13423-014-0661-y}
\item
  Multiple Systems of Perceptual
  Category Learning: Theory and
  Cognitive Tests. \url{https://labs.psych.ucsb.edu/ashby/gregory/sites/labs.psych.ucsb.edu.ashby.gregory/files/pubs/ashbyvalentinhdbkcat_0.pdf}
\item
  The Categorization Experiment:
  Experimental Design and Data Analysis. \url{https://labs.psych.ucsb.edu/ashby/gregory/sites/labs.psych.ucsb.edu.ashby.gregory/files/pubs/ashbyvalentin2018.pdf}
\item
  David Smith prototype v exemplar models
\item
  seminal articles: tversky similarity paper, averaging paper (ashby); medin \& shafer; nosofsky; ashby \& townsend 1986; david smith?
\end{enumerate}

\hypertarget{fnirsneuroimaging}{%
\section{fNIRS/neuroimaging}\label{fnirsneuroimaging}}

\hypertarget{virtualaugmented-reality}{%
\section{Virtual/Augmented Reality}\label{virtualaugmented-reality}}

\hypertarget{statistsical-analysis}{%
\section{Statistsical Analysis}\label{statistsical-analysis}}

\hypertarget{rrmarkdown}{%
\section{R/RMarkdown}\label{rrmarkdown}}

\begin{itemize}
\tightlist
\item
  \href{https://rmd4sci.njtierney.com/}{RMarkdown for Scientists}
\end{itemize}

\hypertarget{productivitygood-research-habits}{%
\section{Productivity/Good Research Habits}\label{productivitygood-research-habits}}

\begin{itemize}
\tightlist
\item
  \href{file:///C:/Users/bohil/Downloads/ResearchHabits.pdf}{Develping good Research Habits}
\item
  especially see the slides labeled ``Reproducability''
\item
  All data edits scripted; all analysis scripted; Graphs \& Tables generated with scripts and automatically pulled into manuscript
\end{itemize}

\hypertarget{intro}{%
\chapter{Data Analysis}\label{intro}}

\hypertarget{analysis-tools}{%
\section{Analysis tools}\label{analysis-tools}}

\hypertarget{jasp}{%
\subsection{JASP}\label{jasp}}

I recommend if you are new to data analysis (and even if you're not) to start with JASP.\\
- \url{https://jasp-stats.org/}
- JASP is freely available software that contains methods for completing most major types of statistical analysis, along with some very sophisticated analysis tools.
- JASP is easy to use
- JASP contains methods for completing ``parametric'' statistical analyses (ANOVA, ttest, regression) which assumes your data are continuous value \& approximately normally distributed. If these assumptions are violated, JASP also contains ``nonparametric'' tests that are equivalents of the parametric versions.
- JASP also does Bayesian versions of all the major analyses (in addition to the traditional ``frequentist'' versions)

\hypertarget{jasp-books}{%
\subsection{JASP books}\label{jasp-books}}

\begin{itemize}
\tightlist
\item
  The JASP website also contains several excellent statistics manuals that are freely-available on their website. These are ideal for beginners.
\item
  \url{https://jasp-stats.org/teaching-with-jasp/}
\item
  In particular see the JASP Manuals section books:
  -- Statistical Analysis in JASP. A Guide for Students by Mark Goss-Sampson (PDF)
  -- Bayesian Inference in JASP: A Guide for Students
\item
  Both of these books provide step-by-step how-to guides to a) choose the correct analysis for your situation (data type \& research question), b) complete the analysis in JASP, and c) how to report the results.
\item
  I recommend you follow the JASP examples and write a summary for each completed analysis right there in the JASP file. When you need to write your results section, just copy/paste these into the document and work on transitions between sections!
  -- Learning Statistics with JASP: A Tutorial for Psychology Students and Other Beginners
\item
  This book provides a very accessible explanation to many of the concepts underlying statistical analysis. The other books mentioned above offer step-by-step how to's, whereas this book provides a deeper explanation. This is essential reading since you'll need to be able to justify any decisions you make about statistical methods used and interpretations of those (which is always a contentious part of peer review).
\end{itemize}

\hypertarget{rrstudiogithub}{%
\subsection{R/RStudio/GitHub}\label{rrstudiogithub}}

One of our goals is to create reproducible research. This means that you could give someone your data and analysis files and they could re-run your analyses and get the identical results.This is possible in programs like JASP, but a scripting language like R is much better for this. I recommend starting with JASP in order to become productive right away, but over time you should learn to use R and carry out much of your analysis by writing code. This is invaluable - even within the lab for sharing or collaborating on projects.

I suggest analyzing your data in JASP, then trying to get R to do the same things (data manipulation, visualization, statistical tests). Over time you should build routines that allow you to do much of what you want to do in R and not worry about using JASP (although I increasingly see a value in using both since JASP is so powerful and easy to use; it serves as a check on your R code).

Your (long-term) goal should ultimately be to complete all data analysis in code, with literally NOTHING done by hand (e.g., data manipulation)!

Start by installing R/RStudio on your computer
- \url{https://rstudio.com/products/rstudio/}

Some great resources for getting started
- \url{https://www.statmethods.net/index.html}
- \url{https://rstudio.com/resources/cheatsheets/}
- Book: R for data science: \url{https://r4ds.had.co.nz/}
- \url{https://www.tidyverse.org/}

Of course - you should take advantage of all the resources already created by other lab members (e.g., the cdlutilities package on github) and all the packages created by others that are freely available (e.g., the tidyverse)

\hypertarget{steps-in-every-analysis}{%
\section{Steps in every analysis}\label{steps-in-every-analysis}}

Every analysis you carry out should include several steps in approximately this order:

\begin{enumerate}
\def\labelenumi{\arabic{enumi})}
\tightlist
\item
  Variable types: What is the scale of your variables? (e.g., continuous, discrete; interval, ordinal, nominal)
\item
  Tidy data: Put your data in tidy format. This usually means 1 observation/row and independent/dependent variables in each column.
\end{enumerate}

\begin{itemize}
\tightlist
\item
  \url{https://vita.had.co.nz/papers/tidy-data.pdf}
\end{itemize}

\begin{enumerate}
\def\labelenumi{\arabic{enumi})}
\setcounter{enumi}{2}
\tightlist
\item
  Visualize: Visualize the data (make plots to see distribution of each variable including outliers, correlation between variables, Q-Q plots to assess normality)
\item
  Descriptive analysis: Carry out descriptive analyses (means, medians, and especially look for violations of normality)
\item
  Question: What question are you trying to answer? Whether groups differ? Whether they are correlated? Whether they differ from 0? See the book by Goss-Sampson (cited above from the JASP website) - it contains an invaluable Appendix with guidance on which test goes with which question, as well as guidance about what to do if normality is violated in your data
\item
  Inferential stats: Carry out the appropriate inferential statistical test. Always make plots within each analysis!
\item
  Interpret results: I recommend you follow the examples in the Goss-Sampson book for each analysis and write a summary of the results right there in the JASP file (just as they do in the JASP example analysis files). If you're working in R, carry out all analysis in an R Markdown (.RMD) file that includes analysis script as well as interpretation text.
\end{enumerate}

See the JASP books cited above for examples and guidance on how to carry out each of these steps

\hypertarget{confirmatory-vs.-exploratory-analysis}{%
\section{Confirmatory vs.~Exploratory analysis}\label{confirmatory-vs.-exploratory-analysis}}

\begin{itemize}
\tightlist
\item
  Confirmatory analysis: Start by testing the hypotheses that led you to design your study as you did. This is probably reflected in the independent and dependent variables you chose to include in the study. Why else would you include a variable in your study if you don't intend to measure it. (Sometimes this does happen - you need to be clear in the writing about why you did that).
\item
  When you carry out confirmatory (i.e., planned) analyses, you have to be conservative in how you carry out your statistical analyses. For example, you shouldn't do anything to your data (e.g., unexpected group splits) before completing the basic analyses and seeing how things worked out (don't short-circuit your analyses half way through if you don't think things appear to be working out; complete all planned analyses first!). If you conduct an ANOVA, your post-hoc comparisons need to be a) limited to what was found significant in an omnibus test and b) corrected for multiple comparisons
\item
  AFTER you have completed your planned (confirmatory) analyses, feel free to do all the exploratory analysis you like! You just have make it clear when you write/present your results that you are switching from confiratory to exploratory analysis. If you do that, then the reader knows to take the exploratory findings with a grain of salt. Exploratory analyses are important and may suggest new hypotheses, but these hypotheses usually require a new study to provide a proper test.
\end{itemize}

\hypertarget{writing}{%
\chapter{Writing}\label{writing}}

Some guidelines and advice on scientific writing.

\hypertarget{general-advice}{%
\section{General advice}\label{general-advice}}

Use R Markdown
* spell checking in .Rmd
* Creating a bibliography using .bib files

\hypertarget{start-writing-early}{%
\section{Start writing early}\label{start-writing-early}}

Write a short outline/draft of the method section when planning an experiment
* as you add details, add this to the outline (can be bullet points at first).
* IMPORTANT: If/When the experiment becomes a reality, write out the complete detailed Method Section right away - don't wait until after data collection is done and data is analyzed. You will remember why things were done better if you write it immediately. Also power analysis should be done for every study and reported here right from the start.

\begin{itemize}
\item
  Introduction/Lit review - same as for methods. Work through the logic of your arguments and enumerate them, and their rationale(s) in an intro section BEFORE you run a study. You can wait to turn this into prose as soon as the experiment becomes a reality (i.e., is running), but then you should write up a draft of the intro/lit review/experiment overview/hypotheses sections and any needed theoretical sections (e.g., plan for statistical analysis of fNIRS data, General Recognition Theory section, etc.).
\item
  You can also put in a lot of the references at this time, and make sure your reference section is building correctly as you go!
\item
  Summarize all PLANNED analyses and write as much of this as you can ahead of time too. This will be more tentative in terms of the writing, but the analysis itself should be completely knowable before starting the experiment.\\
  In fact - your study variables imply what analyses you'll carry out (e.g., what the variables are, the scale of the data will dictate the statistical analyses, modeling analysis planned, ANOVA details)
\end{itemize}

\hypertarget{misc}{%
\subsection{misc}\label{misc}}

\begin{itemize}
\item
  every study can be boiled down to a single page summary of results; we should always create this before writing
\item
  only write after the story is clear - the results are analyzed and evaluated and we've based our conclusions on them in the summary; then write the prose
\item
  list: honors thesis topics we'd like to see (will consider supervising these only)
\end{itemize}

on motivation, time, \& energy
\textgreater{} most important thing will also get the least external motivation -- research/writing
\textgreater\textgreater{} find a time to do this regulary; eg., 9-noon every weekday. no distractions! (phone, email, people).
everythign else you have to do WILL get done, because it HAS TO (nonstop external pressure to complete things): TA duties; class requirements; program requirements; talks/posters
\textgreater{} this is the most insidious threat to your success. if you establish only 1 good habit let it be this: set aside time every day that is for research (whatever phase of the project you are at). that's 15 hours/week. i suggest 1st thing every morning to get it done. then you can do other things with your day and relieves some pressure
\textgreater{} try not to make your problems other peopl'es problems: i.e., if yoiu have a big assigment, let your advisor know, but get in the habit of just planing ahead to avoid letting it disrupt your responsibilities to the lab for research.
\textgreater\textgreater{} even i struggle with this advice from time to time, but it has worked far better for me than anything else over the years
? do you want to know the secret to success in academics? that's it. i just told it to you. not that glamorous huh? but will you take this advice to heart and establish this habit? only you can decide. the biggest killers of productivity (and therefore success in academe) is distraction, time management, and dealing with stress. establishing this habit addresses them all at once. its the best advice i can give you. (I can lead a horse to water but I can't make it drink) make this an iron rule, and you will thrive.

\begin{itemize}
\tightlist
\item
  projects/ideas
  \textgreater{} venn diagram models
  \textgreater{} year 1: they are a complete subset
  \textgreater{} year 4: overlapping subsets
  \textgreater{} never: separate circles
  \textgreater\textgreater{} authorship: using lab resources which i'm ultimately responsible for (this includes my time and effort in advisement); i'm not in favor of you doing something on your own without an advisor; if you want to work on a project with a different advisor, it needs my approval first. look at it from my perspective: trying to run this lab, need people who want to learn from me. if you decide this is not for you, let me know - maybe there's a different advisor. best not to go to them first. if youre not comfortable talking to me about this, talk to the program director (or above if i'm somehow the director at some point)
  \textgreater{} authorship: i'll almost always be corresponding author; maybe not when you're at end of training.
\end{itemize}

\hypertarget{functional-near-infrared-spectroscopy-fnirs}{%
\chapter{Functional Near-Infrared Spectroscopy (fNIRS)}\label{functional-near-infrared-spectroscopy-fnirs}}

\hypertarget{nirsport-88-system-nirx-medical-technologies-inc.}{%
\section{NIRSPort 88 system (NIRX Medical Technologies, Inc.)}\label{nirsport-88-system-nirx-medical-technologies-inc.}}

We have a NIRX nirsport88 mobile system, which no longer appears on the Nirx website.

\begin{figure}
\centering
\includegraphics{docs/resources/nirsport88.jpg}
\caption{NIRx NirSport88 System}
\end{figure}

Nevertheless, everything needed (data acquisition \& analysis software, support, training) can be found on the NIRx website:
\url{https://nirx.net/}

\begin{enumerate}
\def\labelenumi{\arabic{enumi}.}
\tightlist
\item
  download analysis software \& and read manual
\end{enumerate}

\begin{itemize}
\tightlist
\item
  \url{https://nirx.net/software}
\item
  \url{https://nirx.net/nirslab-1}
\item
  NITRC site (where you'll actually download): \url{https://www.nitrc.org/frs/?group_id=651}
\end{itemize}

As you read the manual, try it out with real data (e.g., nirx.net has some data sets you can download if needed).

\begin{enumerate}
\def\labelenumi{\arabic{enumi}.}
\setcounter{enumi}{1}
\tightlist
\item
  read this document on cortical functions for background on localizing Brodmann areas we'll be recording from using the International 10-20 system used in EEG
\end{enumerate}

\begin{itemize}
\tightlist
\item
  \url{https://thebrainstimulator.net/docs/external/Trans_Cranial_Technologies-cortical_functions_ref_v1_0.pdf}
\end{itemize}

Also very helpful: Wikipedia page on Brodmann areas (w. hyperlinks to each area):
\url{https://en.wikipedia.org/wiki/Brodmann_area}
- Provides summary of functions along with references for each brodmann area, and usually an image showing the region

TO ADD:
1. need note on how to examine probe layouts in nirslab

\hypertarget{fnir-devices-imager-1000-from-fnir-devicesbiopac}{%
\section{fNIR Devices Imager 1000 (from fNIR Devices/Biopac)}\label{fnir-devices-imager-1000-from-fnir-devicesbiopac}}

\hypertarget{virtual-augmented-reality}{%
\chapter{Virtual \& Augmented Reality}\label{virtual-augmented-reality}}

In the lab we have the following equipment\ldots{}

\hypertarget{virtual-reality}{%
\section{Virtual Reality}\label{virtual-reality}}

\hypertarget{augmented-reality}{%
\section{Augmented Reality}\label{augmented-reality}}

\hypertarget{students}{%
\chapter{Students}\label{students}}

This chapter contains information for undergraduate research assistants and graduate students. It coverse topcis related to getting started in the lab, paperwork, and program
requirements for grad students.

\hypertarget{ph.d.-students}{%
\section{Ph.D.~Students}\label{ph.d.-students}}

\hypertarget{hfc-program-requirements}{%
\subsection{HFC Program Requirements}\label{hfc-program-requirements}}

Human Factors PhD program webpage (and link to Program Handbook)
\url{https://sciences.ucf.edu/psychology/graduate/ph-d-human-factors-and-cognitive-psychology/}

Milestones for the UCF Human Factors \& Cognitive Psychology Program (for graduate students)

\begin{itemize}
\tightlist
\item
  Research-related class numbers

  \begin{itemize}
  \tightlist
  \item
    PSY 7919 Research
  \item
    PSY 7980 Dissertation
  \end{itemize}
\end{itemize}

\hypertarget{undergraduate-students}{%
\section{Undergraduate Students}\label{undergraduate-students}}

\hypertarget{volunteer-form}{%
\subsection{Volunteer form}\label{volunteer-form}}

THIS SECTION REDUNDANT (IF WE KEEP THE PAPERWORK IN CH 1)

\begin{enumerate}
\def\labelenumi{\arabic{enumi}.}
\tightlist
\item
  Volunteer form
\end{enumerate}

\begin{itemize}
\tightlist
\item
  Before an individual can start his or her volunteer assignment, the department must complete the Volunteer Services Agreement. Go here for details and to access the form(s):
\end{itemize}

\url{https://compliance.ucf.edu/enterprise-risk-management/university-volunteers/}

\begin{enumerate}
\def\labelenumi{\arabic{enumi}.}
\setcounter{enumi}{1}
\tightlist
\item
  If you want your voluneering to appear on your transcript, you must submit a URA form.
\end{enumerate}

\begin{itemize}
\tightlist
\item
  We need to ask the undergraduate advising office for this form
\item
  From Director of Undergraduate Advising Karen Cox: ``The process is departmentally driven. So, the Advising Center helps with the admin side of things for the faculty. To make it easier for you.''
\end{itemize}

``You are welcome to tell students to send the forms to \href{mailto:psychadvising@ucf.edu}{\nolinkurl{psychadvising@ucf.edu}} after they have your approval, fill out their student info, fill in the assignments, and your signature is on the form.''

So it appears you need to e-mail the advising office for the form and attempt to fill it out. Dr.~Bohil can then approve it before you submit it.

\begin{itemize}
\tightlist
\item
  point of contact: Karen Cox, Director of Undergraduate Advising:
  \url{https://sciences.ucf.edu/psychology/people/kox-karen/}
\end{itemize}

QUESTIONS ABOUT COURSE NUMBERS, FORMS or REQUIREMENTS?
Contact the Psychology Advising Center
\url{https://sciences.ucf.edu/psychology/undergraduate/advising/}

COURSE CATALOG \& program information
\url{https://sciences.ucf.edu/psychology/undergraduate/}

\begin{itemize}
\tightlist
\item
  Class numbers (see course catalog for descriptions)

  \begin{itemize}
  \tightlist
  \item
    PSY 4903H Honors Directed Reading I (Independent study)
  \item
    PSY 4904H Honors Directed Reading II (Independent study)
  \item
    PSY 4912 Directed Independent Research (Research)

    \begin{itemize}
    \tightlist
    \item
      Can enroll for 0 or 1 credit hours
    \end{itemize}
  \end{itemize}
\end{itemize}

UNDERGRADUATE RESEARCH
\url{https://our.ucf.edu/current/overview/}

Enroll Students in 0-credit 4912 for the Variable Semester
See our.ucf.edu for enrollment deadline
Working with undergraduate researchers who are not already enrolled in thesis or 4912 hours this semester? Enroll students in 0-credit 4912 for the variable semester by Friday, February 21 (2020). Speak to your advising office for more information on this process.
\url{https://our.ucf.edu/current/courses/}

Summer Undergraduate Research Fellowship (SURF)
Applications due: March 2, 2020
SURF students participate in independent research and creative projects in collaboration with UCF Faculty during the summer. All fellows receive a \$1500 scholarship. Their faculty members receive \$250 supply fund.
\url{https://our.ucf.edu/current/programs/}

OUR Student Research Grants Summer Funding
Proposals due: March 27, 2020
Students can apply for grants to support research costs (\$750 for individual projects or \$1250 for group projects); students traveling to conduct research are eligible for up to \$1250 to support travel (i.e.~flights, hotel).
\url{https://our.ucf.edu/current/programs/}

from chrysalus re HUT course sequence;
The thesis goes like this
Directed Readings I (required; 3 credits): students work on their literature review and proposal for their study. If they are not able to finish their proposal in this semester they can register for Directed Readings II.
Directed Readings II (options; 1-3 credits): student finishes the proposal if they were not able to do so in directed readings I
Undergraduate Thesis I (required; 3 credits): students should work on collecting any needed data, analyzing data, writing up results and finish the entire thesis. They finish this with a thesis defense. If they are not able to do this in one semester they can register for Undergraduate Thesis II
Undergraduate Thesis II (optional; 1-3 credits): students finish their entire thesis if they were not able to do so in undergraduate thesis I.

\hypertarget{psychopy}{%
\chapter{Psychopy for Data Collection}\label{psychopy}}

\href{https://www.psychopy.org}{Psychopy} is a free open-source package for building experiments in Python that is developed and maintained by the University of Nottingham in the UK. It has been active since 2007, but recent updates have introduced a purpose-built solution for online data collection called \href{https://pavlovia.org/\#about}{Pavlovia}. Pavlovia is a front-end for gitlab and allows for version control of experiment code.

\hypertarget{getting-started}{%
\section{Getting Started}\label{getting-started}}

\hypertarget{psychopy-1}{%
\subsection{PsychoPy}\label{psychopy-1}}

A standalone version (no Python install required) can be downloaded from the \href{https://www.psychopy.org/download.html}{project's website}. After installing Psychopy, the best place to start is with the \href{https://www.psychopy.org/documentation.html}{PsychoPy documentation}, which includes guides on getting started building experiments. Included with the initial download are several minimalist, functional experiment demos that provide a good starting point for building your experiment.

\hypertarget{pavlovia}{%
\subsection{Pavlovia}\label{pavlovia}}

After installing PsychoPy, the next step in getting an online experiment running is \href{https://gitlab.pavlovia.org/users/sign_in}{setting up an account on Pavlovia}. \textbf{Note}: In order for the account to fall under the UCF site license,
the associated email must have the *@ucf.edu domain. Best practice is to use the email of the faculty member in charge of the lab. This ensures that the faculty member retains admin control of the account. The account username and password will likely need to be provided to students to allow them to develop and access experiments on Pavlovia, and the username and password need not correspond to the faculty email or any existing passwords. Notification settings can also be changed on \href{https://gitlab.pavlovia.org/profile/emails}{Pavlovia} to control where the various types of email notifications are directed.

\hypertarget{sona}{%
\subsection{SONA}\label{sona}}

\hypertarget{frequently-asked-questions-faqs}{%
\section{Frequently Asked Questions (FAQs)}\label{frequently-asked-questions-faqs}}

While many issues one might encounter are addressed in the \href{https://www.psychopy.org/documentation.html}{documentation}, keep in mind that, as with all free open-source software, PsychoPy and Pavlovia are perpetually under development and there will be bugs. Multiple resources exist to facilitate solving such problems -- the \href{https://discourse.psychopy.org/}{PsychoPy Forum} is home to a growing and very helpful community, UCF Psychology
will be adding a private discussion board (to be added later) specifically for helping each other use PsychoPy -- but ultimately the how-to of building your experiment falls to you, the researcher. A few common questions have, however, been addressed below, and suggestions for addition to this list are always welcomed.

\hypertarget{compliance}{%
\subsection{Compliance}\label{compliance}}

\begin{enumerate}
\def\labelenumi{(\arabic{enumi})}
\tightlist
\item
  Can I collect restricted data using PsychoPy?
\end{enumerate}

No.~While the program itself allows for the collection of many kinds of data, including survey, the data storage of Pavlovia does not meet UCF's InfoSec requirements for sensitive or identifiable data.\\
~\\

\begin{enumerate}
\def\labelenumi{(\arabic{enumi})}
\setcounter{enumi}{1}
\tightlist
\item
  What can I do if I need to collect restricted data then?
\end{enumerate}

Additional questions can be added to SONA's intake questionnaire to securely collect sensitive or identifiable data.

\hypertarget{technical}{%
\subsection{Technical}\label{technical}}

\begin{enumerate}
\def\labelenumi{(\arabic{enumi})}
\setcounter{enumi}{2}
\tightlist
\item
  Is there a built-in component for providing feedback based on participants' responses?
\end{enumerate}

Astonishingly, no. The way to provide corrective feedback is using the \href{https://www.psychopy.org/builder/components/code.html}{Code Component}. An example of the Python code necessary is provided below. Note that the strings used are f-strings -- this is necessary for successful translation to JS.

In Begin Experiment:\\
\texttt{msg=\textquotesingle{}\textquotesingle{}\ \ \#msg\ variable\ just\ needs\ some\ value\ at\ start}~\\

In Begin Routine:\\
\texttt{if\ not\ resp.keys:\ \#\ failed\ to\ respond}~\\
\hspace*{0.333em} \texttt{msg="Repeat\ trial.\ Use\ the\ labelled\ keys."}\\
\texttt{elif\ resp.corr:\ \#stored\ on\ last\ run\ routine}~\\
\hspace*{0.333em} \texttt{msg=f"Correct!\ \textbackslash{}n\textbackslash{}nAnswer:\ \{AnswerName\}"}\\
\texttt{elif\ not\ resp.corr:\ \#wrong\ answer}~\\
\hspace*{0.333em} \texttt{msg=f"Wrong.\ \textbackslash{}n\textbackslash{}nAnswer:\ \{AnswerName\}"}\\

In End Routine: (If repeating trials without responses)\\
\texttt{if\ not\ resp.keys:}~\\
\hspace*{0.333em} \texttt{repeatTrial.finished\ =\ False\ \#\ "repeatTrial"\ must\ be\ "trials"\ in\ the\ JS\ code}\\
\texttt{else:}~\\
\hspace*{0.333em} \texttt{repeatTrial.finished\ =\ True}\\
~\\

\begin{enumerate}
\def\labelenumi{(\arabic{enumi})}
\setcounter{enumi}{3}
\tightlist
\item
  Why does my feedback work in Python but not in JS?
\end{enumerate}

The built-in Python-\textgreater JS translator is good but not perfect. Be sure you are using f-strings

\begin{enumerate}
\def\labelenumi{(\arabic{enumi})}
\setcounter{enumi}{4}
\tightlist
\item
  Why are my stimulus image files not being read into my experiment for presentation?
\end{enumerate}

PsychoPy's pathing is picky - be sure your image files (.PNG, .wav, etc.) are in the root directory with your experiment file (.psyexp or .py) and your conditions file(s) (.csv, .xlsx, etc.)

\hypertarget{miscellaneous}{%
\subsection{Miscellaneous}\label{miscellaneous}}

\hypertarget{sample_paper}{%
\chapter{Sample paper}\label{sample_paper}}

This paper is an example for demonstrating all the usual steps in creating a paper - in particular the results section. The goal will be to demonstrate (and develop) tools needed to a) format data, b) visualize trends in data, c) perform statistical analyses, and d) perform some more specialized analyses (e.g., computational modeling, fNIRS analysis).

\hypertarget{road-map}{%
\section{Road map}\label{road-map}}

\begin{enumerate}
\def\labelenumi{\arabic{enumi}.}
\tightlist
\item
  Introduction/literature review section should list:

  \begin{enumerate}
  \def\labelenumii{\roman{enumii})}
  \tightlist
  \item
    Study design, including independent variables (and their levels \textbf{and measurement scale}), dependent variables, and covariates (additional measures to either control for or examine correlation with primary variables)
  \item
    Each hypothesis and the basis for it (e.g., theory ,literature, our previous papers, etc.)
  \item
    Clear DIRECTIONAL predictions for variables included in the current research.\footnote{Directional means our predictions should be specific enough to say, e.g., Factor A level 1 should be \textgreater{} level 2. It is rarely the case that we should state a non-directional hypothesis (e.g., Factor A level 1 will differ from level 2). What a lame ``hypothesis''. If we don't have a basis for some expectation, then why are we spending time and resources to carry out this research? Of course, in the context of a research project there might be included a variable that is exploratory and for which we will ``see if there's a difference'', but in that case we don't really have anything resembling a ``hypothesis'', and it alone wouldn't form a sufficient basis for a study. (Yes, some research is not hypothesis driven but again that will be very rare in our lab.)}
  \end{enumerate}
\item
  Methods
\item
  Results (confirmatory {[}i.e., planned{]} analysis first, then exploratory analysis if any)

  \begin{enumerate}
  \def\labelenumii{\roman{enumii})}
  \tightlist
  \item
    Analysis 1 (e.g., accuracy rates for each group; ANOVA main effect). Analysis 1 result. Hypothesis supported or not?
  \item
    Analysis 2 (e.g., Factor 2 main effect of levels?). Analysis 2 result. Implications for hypotheses?
  \item
    Etc. for each planned type of analyis for each variable (planned analysis, result, status of hypothesis based on this result)
  \item
    Describe any exploratory analysis conducted, findings, and implications\footnote{Note the implications for hypotheses will not really tend to be elaborated on in depth in the results section, although they might be briefly mentioned. E.g., As predicted, accuracy improved more rapidly over blocks in group 1 than group 2.}
  \end{enumerate}
\item
  Discussion/Conclusions. Summarize (in list form) your conclusions about the patterns found in the data and your conclusions about all hypotheses based on these data. Then consider any exploratory analysis conducted and its implications.
\end{enumerate}

\hypertarget{project-road-map}{%
\subsection{Project road map}\label{project-road-map}}

\begin{enumerate}
\def\labelenumi{\arabic{enumi}.}
\tightlist
\item
  Intro/lit review
\item
  Methods
\item
  Results
\item
  Discussion/Conclusions
\end{enumerate}

\hypertarget{introlit-review}{%
\section{Intro/lit review}\label{introlit-review}}

\hypertarget{methods}{%
\section{Methods}\label{methods}}

\hypertarget{results}{%
\section{Results}\label{results}}

\hypertarget{experiment-1}{%
\subsection{Experiment 1}\label{experiment-1}}

\hypertarget{data-preprocessing}{%
\subsubsection{Data preprocessing}\label{data-preprocessing}}

\href{https://r4ds.had.co.nz/data-import.html}{Read in data} and organize for analysis. Data is usually stored in \href{https://r4ds.had.co.nz/tidy-data.html}{``Tidy'' format}, or you should put it into Tidy format after reading it in. This makes data easy to view and modify.

However, Tidy format usually means the data are in Wide format. This is easy to work with and easy to examine. But for plotting and statistical analysis, you usually need the data in \textbf{long format}. So after reading in data and arranging into Tidy, wide format, make a copy of the data in long format to be used in visualization and statistics.

Here are some examples of converting from wide to long format: \href{https://www.theanalysisfactor.com/wide-and-long-data/}{{[}1{]}}, \href{http://www.cookbook-r.com/Manipulating_data/Converting_data_between_wide_and_long_format/}{{[}2{]}}; to see how to do this using the tidyverse, see \href{https://r4ds.had.co.nz/tidy-data.html\#pivoting}{section 12.3 Pivoting} in \href{https://r4ds.had.co.nz/}{R for Data Science}

None of this will appear in the manuscript of course. So all the R code chunks should include the tag `include = FALSE', meaning the code and its output don't show up in the knitted version of the paper.

Note on lists: I recommend that for each experiment in the paper, you create a separate ``list'' object to store all the data and output specific to that experiment. If there is an experiment 2 reported, you may want to re-run a lot of the same analyses in both experiments. If you keep the same variable names across experiments, then earlier values get overwritten by later ones, which could create a lot of problems. After all analyses have been conducted, you should be able to examine the input and output values for each experiment reported.

\hypertarget{visualization}{%
\subsubsection{Visualization}\label{visualization}}

\hypertarget{statistical-analyiss}{%
\subsubsection{Statistical Analyiss}\label{statistical-analyiss}}

\hypertarget{discussionconclusions}{%
\section{Discussion/Conclusions}\label{discussionconclusions}}

  \bibliography{book.bib,packages.bib}

\end{document}
